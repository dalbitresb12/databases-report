\documentclass[../main.tex]{subfiles}

\graphicspath{{\subfix{../images/}}}

\begin{document}

\section{Empresa seleccionada}

\subsection{Datos generales}

\begin{itemize}
  \item Razón social: Coursera Inc.
  \item Página web: \url{https://coursera.org/}
  \item Tipo de empresa: \textit{Incorporated} (Corporación)
  \item Condición: Activa
\end{itemize}

Para la finalidad de este trabajo, seleccionamos la empresa
internacional Coursera Inc. Es, desde octubre de 2011, una
plataforma de educación virtual que ofrece diferentes cursos
y programas en línea. Es una herramienta web desarrollada por
la Universidad de Stanford y todos sus cursos están destinados
a cualquier interesado, sea cual sea su nivel académico.

\subsection{Misión y Visión}

\begin{itemize}
  \item \textbf{Misión:} \newline
        Ofrecer a cualquier persona, en cualquier lugar, acceso
        a cursos y títulos en línea de las principales
        universidades y empresas.
  \item \textbf{Visión:} \newline
        Proporcionar experiencias de aprendizaje que transforman
        la vida de los estudiantes de todo el mundo.
\end{itemize}

\subsection{Procesos principales}

\begin{enumerate}
  \item Administración de contenido: \newline
        El proceso mediante el cual las entidades afiliadas,
        en su mayoría universidades, recopilan y ordenan el
        material de aprendizaje antes de su publicación en
        la plataforma y Coursera verifica su calidad. Coursera
        está asociada con más de 200 universidades y empresas
        líderes en el mundo, ofreciendo aprendizaje digital
        en gran variedad de modalidades según la demanda de
        sus usuarios.
  \item Oferta de diferentes categorías de aprendizaje: \newline
        Coursera ofrece una gran variedad de modalidades de
        estudio, como son a corto plazo (proyectos guiados,
        cursos y especializaciones) y a largo plazo (certificados
        profesionales, MasterTracks, bachilleratos y maestrías).
  \item Afiliación con entidades: \newline
        Proceso por el cual Coursera ofrece sus servicios y
        plataforma a universidades y empresas. De esta manera,
        estas instituciones y sus asociados (estudiantes y
        trabajadores) tienen acceso a material especial que
        deseen aprender en las modalidades especificadas.
\end{enumerate}

\subsection{Problemas detectados}

\begin{itemize}
  \item El principal problema de Coursera son las trampas y el
        plagio, las cuales originan diversas quejas entre los
        estudiantes y una menor credibilidad de sus cursos para
        obtener un buen aprendizaje.
  \item La falta de atención a los estudiantes de parte de los
        docentes también es un problema, debido a que varias
        quejas en los foros de Coursera no son respondidas.
\end{itemize}

\subsection{Reglas de negocio}

\begin{enumeratedtable}[prefix=RN]
  \item{La página web debe estar disponible las 24 horas del día.}
  \item{El estudiante puede adquirir varios cursos.}
  \item{El estudiante puede realizar el pago a través de una tarjeta de crédito, débito o PayPal.}
  \item{El estudiante puede darle una calificación al curso.}
  \item{El estudiante puede participar en los foros de debate del curso que está tomando.}
  \item{El estudiante puede ver la lista de cursos que ha completado.}
\end{enumeratedtable}

\end{document}
